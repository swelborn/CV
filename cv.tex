%-------------------------
% Rover Resume - Base Template
% Link: https://github.com/subidit/rover-resume
%
% Shows code for various formatting options for different resume sections.
% Education and Projects have single-line headers; while Experience uses double-line.
% Some formatting codes are kept inline; consider \newcommand{cmd}{def}.
% Excludes hyperref and icons for readability; MVP version.
% Explore other templates for more options.
% Mix and match as desired. Be consistent with headers and sub-headers.
%------------------------

\documentclass[11pt]{article} % fontsize 10pt/11pt/12pt

\usepackage[margin=1in, letterpaper]{geometry}
\setcounter{secnumdepth}{0} % remove section numbering
\usepackage{titlesec}
\titlespacing{\subsection}{0pt}{*0}{*0} % remove vertical spacing above and below
\titlespacing{\subsubsection}{0pt}{*0}{*0}
\titleformat{\section}{\large\bfseries\uppercase}{}{}{}[\titlerule]
\titleformat*{\subsubsection}{\large\itshape}
\usepackage{enumitem}
\setlist[itemize]{noitemsep,left=0pt .. \parindent}
\pagestyle{empty} % remove page number
\pdfgentounicode=1

\usepackage{array}
\usepackage{booktabs}
\usepackage{colortbl}
\usepackage[dvipsnames]{xcolor}
\usepackage{hyperref}
\usepackage{cleveref}
\usepackage{doi}
\usepackage{fontawesome5}
\usepackage{soul}

\definecolor{verylightgray}{gray}{0.85}
\hypersetup{
  colorlinks=true,
  linkcolor=PineGreen,
  filecolor=magenta,
  urlcolor=NavyBlue,
  citecolor=black
}

\usepackage[backend=biber, sorting=none, style=numeric, maxbibnames=10, giveninits=true, defernumbers=true]{biblatex}
\addbibresource{references.bib}
\addbibresource{conferences.bib}
\AtEveryCitekey{\clearfield{eprint}}

\usepackage{xstring}
\usepackage{etoolbox}
\newboolean{bold}
\newcommand{\makeauthorsbold}[1]{%
  \DeclareNameFormat{author}{%
    \setboolean{bold}{false}%
    \renewcommand{\do}[1]{\expandafter\ifstrequal\expandafter{\namepartfamily}{####1}{\setboolean{bold}{true}}{}}%
    \docsvlist{#1}%
    \ifthenelse{\value{listcount}=1}
    {%
      {\expandafter\ifthenelse{\boolean{bold}}{\mkbibbold{\namepartfamily\addcomma\addspace \namepartgiveni}}{\namepartfamily\addcomma\addspace \namepartgiveni}}%
    }{\ifnumless{\value{listcount}}{\value{liststop}}
      {\expandafter\ifthenelse{\boolean{bold}}{\mkbibbold{\addcomma\addspace \namepartfamily\addcomma\addspace \namepartgiveni}}{\addcomma\addspace \namepartfamily\addcomma\addspace \namepartgiveni}}%
      {\expandafter\ifthenelse{\boolean{bold}}{\mkbibbold{\addcomma\addspace \namepartfamily\addcomma\addspace \namepartgiveni\addcomma\isdot}}{\addcomma\addspace \namepartfamily\addcomma\addspace \namepartgiveni\addcomma\isdot}}%
    }
    \ifthenelse{\value{listcount}<\value{liststop}}
    {\addcomma\space}{}
  }
}
\makeauthorsbold{Welborn,Samuel S}

\begin{document}

\begin{center}
  \begin{minipage}{0.5\textwidth}
    {\LARGE\bfseries
      SAMUEL S. WELBORN
    } \\ \medskip
    Curriculum Vitae
  \end{minipage} \hfill
  \begin{minipage}{0.4\textwidth}
    \raggedleft
    {\faEnvelope} \ {swelborn@lbl.gov} \\
    \href{https://www.linkedin.com/in/swelborn}{\faLinkedin} \ linkedin.com/in/swelborn \\
    \href{https://github.com/swelborn}{\faGithub} \ github.com/swelborn \\
    \href{https://scholar.google.com/citations?user=WsQfglgAAAAJ&hl=en&authuser=1}{\faGraduationCap} \ Google Scholar
  \end{minipage}
\end{center}

%=================%
\section{Education \& Training}
%=================%
\setlist[itemize]{noitemsep}
\subsection{National Energy Research Scientific Computing Center (NERSC)}
\begin{itemize}
  \item NERSC Science Acceleration Program (NESAP) Postdoctoral Fellow \hfill 2022--Present
  \item Supervisors: Dr.\ Deborah J.\ Bard, Dr.\ Shane Canon
\end{itemize}

\subsection{SLAC National Accelerator Laboratory}
\begin{itemize}
  \item DOE-SCGSR Fellow \hfill 2021--2022
  \item Advisor: Dr.\ Johanna Nelson Weker
\end{itemize}

\subsection{University of Pennsylvania}
\begin{itemize}
  \item PhD in Materials Science and Engineering \hfill 2022
  \item Thesis: `X-ray Scattering Investigations into Nanoporous Gold's Kinetic Behavior During Dealloying and Coarsening for Applications in 3D Energy Storage'
  \item Advisor: Prof.\ Eric Detsi
\end{itemize}

\subsection{Virginia Polytechnic Institute and State University }
\begin{itemize}
  \item B.S. Chemical Engineering, \textit{summa cum laude} \hfill 2016
  \item B.A. Chemistry, \textit{summa cum laude} \hfill 2016
\end{itemize}

%=================%
\section{Postdoctoral Experience}
%=================%
\setlist[itemize]{noitemsep,left=0pt .. \parindent}

\subsection{Data Science Engagement Group (DSEG), NERSC}

\subsubsection{NESAP for Data Postdoctoral Fellow \hfill Nov 2022--Present}

\begin{itemize}
  \item Streaming from the National Center for Electron Microscopy (\href{https://foundry.lbl.gov/about/facilities/the-national-center-for-electron-microscopy-ncem/}{NCEM}) to NERSC
    \begin{itemize}
      \item Identified file I/O bottleneck on the 4D Camera detector, which generates data at 480 Gbit/s.
      \item Implemented \textit{\href{https://zeromq.org/}{ZeroMQ}}-based pipeline to stream data over 100 GbE directly to Perlmutter compute nodes, boosting raw throughput by up to \textbf{14x}.
      \item Enhanced \textit{\href{https://github.com/OpenChemistry/distiller}{Distiller}}, NCEM's web portal, to enable non-HPC experts to initiate and control streaming services at NERSC through the \href{https://docs.nersc.gov/services/sfapi/}{Superfacility API}.\
      \item Published two manuscripts on this work: \href{https://link.springer.com/chapter/10.1007/978-3-031-73716-9_17}{ISC 2024} and \href{https://doi.org/10.1093/mam/ozae109}{Microscopy and Microanalysis}.
      \item Demonstrated the workflow \textit{live} at SC23 at the DOE Booth (\href{https://github.com/user-attachments/assets/4e6b2dd4-da00-4472-a519-34f49085c1ea}{Computing Sciences photo}).
      \item Developing a flow-based programming application for interactive streaming workflows:\\\href{https://github.com/NERSC/interactEM}{interactEM}.
    \end{itemize}

  \item Streaming from the Advanced Light Source (\href{https://als.lbl.gov/}{ALS}) to NERSC
    \begin{itemize}
      \item Gathered requirements from tomography and ptychography beamlines at ALS.\
      \item Enabled real-time tomography data reconstruction pipeline that reduces turnaround time from \textgreater 10 minutes to \textbf{\textless 10 seconds}.
      \item Collaborated with the ALS Computing Group to integrate the streaming reconstruction workflow into their existing Prefect tooling for use in production.
    \end{itemize}
    \newpage

  \item NERSC-centered Activities
    \begin{itemize}
      \item Instructed and mentored new users of NERSC's Kubernetes cluster, Spin, in multiple training events.
      \item Ensured NCEM's continued operations through a Perlmutter scheduled maintenance by piloting the Perlmutter On Demand (POD) service. This effort served as a major proof-of-concept for POD.\
      \item Active member of the Streaming Working Group, meeting bi-weekly to discuss streaming efforts with colleagues at NERSC and ESNet.
    \end{itemize}

  \item Professional Development
    \begin{itemize}
      \item Accepted to and participated in the two-week Argonne Training Program on Extreme-Scale Computing (ATPESC) in 2023.
    \end{itemize}
\end{itemize}

%=================%
\section{Technical Skills}
%=================%

\begin{itemize}
  \item \textbf{Expertise:} data streaming and real-time data processing, full-stack development, user interface development (\href{https://onlinelibrary.wiley.com/iucr/doi/10.1107/S1600577524003989}{TomoPyUI}, \href{https://github.com/OpenChemistry/distiller}{Distiller}), synchrotron X-ray techniques (microscopy, tomography, spectroscopy, diffraction, scattering)
  \item \textbf{Programming languages:} Python, C++, TypeScript
  \item \textbf{Technologies used:} ZeroMQ, NATS, git, pvaPy, CMake, vcpkg, pydantic, MessagePack, React, Redux, FastAPI, PostgreSQL, Alembic, Kubernetes, Jupyter, Helm, Helmfile, Docker, podman, GitHub Actions, slurm, Spin (NERSC Kubernetes cluster), Dask, ipywidgets, Superfacility API
  \item \textbf{Professional skills:} Public speaking (14 national and international scientific conference talks, 3 invited talks, panelist at ISC 2024 workshop), federal grant writing for scientific funding
\end{itemize}

%=================%
\section{Graduate Research Experience}
%=================%

\subsection{Nelson Weker Group: SLAC National Accelerator Laboratory \hspace*{\fill} 2021--2022}
\begin{flushleft}
  \textit{Fellowship:} Office of Science Graduate Student Research Fellowship (DOE-SCGSR, \$39,000)
\end{flushleft}
\begin{itemize}
  \item X-ray absorption studies on the impact of curvature on charge storage behavior in 3D aperiodic nanoporous battery electrodes
    \begin{itemize}
      \item Developed atomic layer deposition recipe for coating multicomponent cathode (LiMn\textsubscript{2}O\textsubscript{4}) on aperiodic 3D nanoporous scaffolds
      \item Operated multiple X-ray beamlines at the Stanford Synchrotron Radiation Lightsource (SSRL) to study novel energy storage materials
      \item Developed software to align and reconstruct X-ray nanotomography data: \href{https://onlinelibrary.wiley.com/iucr/doi/10.1107/S1600577524003989}{TomoPyUI}
    \end{itemize}

\end{itemize}

\subsection{Detsi Group: University of Pennsylvania \hspace*{\fill} 2016--2022}
\begin{flushleft}
  \textit{Fellowship:} Vagelos Institute for Energy Science and Technology Fellowship (VIEST, \$58,000)
\end{flushleft}
\begin{itemize}
  \item X-ray scattering studies on the kinetic behavior of aperiodic nanoporous materials in real time during electrochemical and thermal processing
    \begin{itemize}
      \item Studied morphological evolution of nanoporous gold during synthesis and thermal coarsening using small- and wide-angle X-ray scattering
      \item Developed a suite of MATLAB and Mathematica programs to model and post-process the corresponding X-ray scattering and electrochemical data
    \end{itemize}
\end{itemize}

\begin{itemize}
  \item Development of three-dimensional tricontinuous bulk conductor-insulator-conductor nanocomposites for high-rate electrical energy storage
    \begin{itemize}
      \item Developed clean room fabrication protocol to make 3D nanoporous metal scaffolds and grow dissimilar layers inside their void space to create bulk conductor-insulator-conductor nanocomposites
      \item Characterized the 3D nanocomposites using electrochemical characterization techniques including cyclic voltammetry and electrochemical impedance spectroscopy
    \end{itemize}
\end{itemize}

%=================%
\section*{Publications}
%=================%
% \subsection*{Preprints}s
\begin{refsection}
  \nocite{*}
  \printbibliography[heading=none, type=misc]
\end{refsection}

\subsection{Journal Articles}
Summary: \textbf{7} first author, \textbf{12} co-author, \textbf{400+} citations (\href{https://scholar.google.com/citations?user=WsQfglgAAAAJ&hl=en&authuser=1}{Google Scholar})

\begin{refsection}
  \nocite{*}
  \printbibliography[heading=none, type=article]
\end{refsection}

%=================%
\section{Conferences and Seminars}
%=================%
\arrayrulecolor{black} % Set the color of the horizontal lines
\setlength{\arrayrulewidth}{1mm} % Set the width of the outer border
\renewcommand{\arraystretch}{1.3} % Increase row height

\arrayrulecolor{black} % Set the color of the lines
\begin{center}
  \begin{tabular}{@{\extracolsep{0pt}}>{\centering\arraybackslash}m{3cm}>{\centering\arraybackslash}m{3cm}>{\centering\arraybackslash}m{5.5cm}>{\centering\arraybackslash}m{2.5cm}}
    \rowcolor{white} \textbf{Date(s)} & \textbf{Location} & \textbf{Conference/Seminar} & \textbf{Contribution} \\
    \cmidrule(lr){1-1}\cmidrule(lr){2-2}\cmidrule(lr){3-3}\cmidrule(l){4-4}
    Oct 15--16, 2025 & Philadelphia, PA & Future Labs Live USA & Invited Oral \\
    \rowcolor{verylightgray}June 10--13, 2025 & Hamburg, Germany & ISC High Performance 2025 & Oral \\
    Nov 17--22, 2024 & Atlanta, GA & SC24: Streaming Birds of a Feather & Oral \& Panelist \\
    \rowcolor{verylightgray}May 12--16, 2024 & Hamburg, Germany & ISC High Performance 2024 & Oral \& Panelist \\
    Oct 22--24, 2024 & Berkeley, CA & NERSC 50th Anniversary User Group Meeting & Poster \\
    \rowcolor{verylightgray}Feb 21--22, 2024 & Berkeley, CA & NERSC Data Day & Oral \\
    Nov 12--17, 2023 & Denver, CO & SC23: DOE Booth & Live Demo \\
    \rowcolor{verylightgray}Sep 11--15, 2023 & Busan, South Korea & 20th International Microscopy Congress & Oral \& Poster\\
    Jun 6, 2022 & (Virtual) & APS Scientific Computation Seminar Series & Invited Oral \\
    \rowcolor{verylightgray}Oct 12, 2021 & (Virtual) & Nanoporous Metals by Alloy Corrosion Symposium & Invited Oral \\
    Oct 10--14, 2021 & Orlando, FL (Virtual) & ECS Fall Meeting & Oral \\
    \rowcolor{verylightgray}Apr 15, 2021 & Philadelphia, PA (Virtual) & Penn MSE Departmental Seminar & Invited Oral \\
    Oct 4--9, 2020 & Honolulu, HI (Virtual) & ECS PRiME 2020 & Oral \\
    \rowcolor{verylightgray}Jun 12, 2020 & Philadelphia, PA (Virtual) & Dual-source and Environmental X-ray Scattering Facility Seminar & Oral \\
    May 21, 2020 & Philadelphia, PA (Virtual) & Philadelphia Regional ECS Graduate Student Seminar Series & Oral \\
    \rowcolor{verylightgray}Dec 1--6, 2019 & Boston, MA & MRS 2019 & Oral \\
    Aug 22--23, 2019 & Dearborn, MI & MetFoam 2019 & Oral \\
    \rowcolor{verylightgray}Aug 19--22, 2019 & Los Angeles, CA & AAAFM 2019 & Oral \\
    Mar 10--14, 2019 & San Antonio, TX & TMS 2019 & Oral \\
    \rowcolor{verylightgray}Feb 24--28, 2019 & Philadelphia, PA & ISNM 2019 & Oral \& Poster \\
    Nov 25--30, 2018 & Boston, MA & MRS 2018 & Oral \\
    \rowcolor{verylightgray}Jun 24--29, 2018 & Hong Kong & NANO 2018 & Poster \\
  \end{tabular}
\end{center}
\newpage

% %=================%
% \section{Conferences and Seminars}
% %=================%

% \newrefcontext
% \begin{refsection}
%   \nocite{*}
%   \printbibliography[heading=none, type=unpublished]
% \end{refsection}

%=================%
\section*{Awards and Distinctions}
%=================%

\subsection*{Graduate}
\begin{itemize}
  \item 2021 DOE Office of Science SCGSR Fellow (Full Stipend, 1 year) \hfill \$39,000
  \item 2020 NNCI Plenty of Beauty at the Bottom image contest \hfill \$1,000
  \item 2019 VIEST Graduate Student Fellow (Full Tuition + Stipend, 1 year) \hfill \$58,000
  \item TMS MetFoam 2019 Registration Award \hfill \$370
\end{itemize}

\subsection*{Undergraduate}
\begin{itemize}
  \item 2016 HyperCube Scholar Award, Virginia Tech Chemistry Department
  \item 2016 Phi Kappa Phi Graduate Fellowship \hfill \$500
  \item 2016 Inducted into Phi Kappa Phi and Phi Beta Kappa Honor Societies
  \item 2015 Accepted into study abroad program at Ruhr-Universität Bochum
  \item 2015 Recipient of Julius P. Bilisoly Scholarship, Virginia Tech Chemistry \hfill \$1,700
  \item 2015 Recipient of Gerhard H. Beyer Chemical Engineering Scholarship \hfill \$2,000
  \item 2014 Recipient of Steven Reese, R.H. Bogle Chemical Engineering Scholarships \hfill \$2,700
  \item 2014 Recipient of Chemistry Summer Research Scholarship \hfill \$5,000
  \item 2014 Academic Excellence Award --- Chemistry Department at Virginia Tech
\end{itemize}

%=================%
\section{Mentoring \& Outreach}
%=================%
\subsection{Mentoring}
\begin{itemize}
  \item Summer 2025: Mentored two NERSC interns for work on \href{https://github.com/NERSC/interactEM}{interactEM}. One intern contributed an RDMA-based messaging service, and another contributed a \textit{Grafana} monitoring dashboard.
  \item Spring-Summer 2020: Mentored a \href{https://www.viper.upenn.edu/}{VIPER} undergraduate and master's student on a simulation project during the COVID lockdown
  \item Fall 2019: Mentored two students before and during Materials Research Society (\href{https://www.mrs.org/}{MRS}) 2019 \& Exhibit in collaboration with NSF's Partnerships for Research and Education in Materials (\href{https://new.nsf.gov/funding/opportunities/prem-partnerships-research-education-materials}{PREM}) program
  \item Fall 2019: Mentored a master's student and two undergraduate Materials Science and Engineering (MSE) students
  \item Spring 2019: Mentored an MSE undergraduate student
  \item Summer 2018--2019: Mentored \href{https://www.viper.upenn.edu/}{VIPER} undergraduate students
\end{itemize}

\subsection{Outreach}
\begin{itemize}
  \item Fall 2019: Developed and conducted electrochemistry demos at \href{https://www.nano.upenn.edu/nanoday/}{NanoDay@Penn}
  \item Summer 2019: Demonstrated electrochemical energy storage systems for the annual Middle School Science Outreach Program at Penn supported by the NSF-MRSEC program.
  \item 2019--2022: Helped recruit new PhD students at Materials Science and Engineering open houses
  \item 2019--2021: Served as The Electrochemical Society Student Chapter Secretary
  \item Fall 2019: Co-founded The Electrochemical Society UPenn student chapter
  \item 2017--2018: Served as the President of the MSE Materials Graduate Organization (\href{https://mse.seas.upenn.edu/graduate-groups-resources/}{MatGO})
  \item 2015--2016: Co-Produced \href{https://www.facebook.com/rocktheblocks/}{Rock the Blocks Music and Arts Festival}
  \item 2012--2016: Served in leadership roles in The Environmental Coalition at Virginia Tech
\end{itemize}

\end{document}
